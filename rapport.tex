\documentclass[]{article}

\usepackage[french]{babel} % choix de la langue : anglais par défaut
\usepackage[utf8]{inputenc} % choix de l'encodage des caractères : utile pour les accents

%%%%%%%%%%%%%%%%%%%%%%%%%%%%%%%%%%%%%%%%%%%%%%%%%%%%
% Math : les packages classiques : 
\usepackage{amsmath} 
\usepackage{amssymb} 

%%%%%%%%%%%%%%%%%%%%%%%%%%%%%%%%%%%%%%%%%%%%%%%%%%%%
% Algo : 
\usepackage[french]{algorithm2e} 

%%%%%%%%%%%%%%%%%%%%%%%%%%%%%%%%%%%%%%%%%%%%%%%%%%%%
% graphique
% \usepackage{graphicx} 


%%%%%%%%%%%%%%%%%%%%%%%%%%%%%%%%%%%%%%%%%%%%%%%%%%%%%%%%%%%
% DOCUMENT

\title{Simulation d’exécution de programmes parallèles}
\author{Olga Pigareva} % optionnel

% \date est optionnel : par défaut, date du jour de compilation
% \date{aujourd'hui} % date fixe
% \date{} % pas de date

\begin{document}

\maketitle % indispensable

\tableofcontents % facultatif

\newpage % permet de changer de page

\section{Motivation}
\subsection{Problème}
On s'intéresse ici de l'exécution de programmes parallèles. Quand on a plusieurs activités, les interblocages peuvent survenir.
Exemple.. un processus peut attendre la fin d'autre et l'inverse, alors le programme reste bloquer.
Pour prévenir l'interblocage, l'idée est de pouvoir déterminer s'il y a l'interblocage lors de la compilation d'un code.
plus précisément on s'intéresse à créer un compilateur simple qui va reconnaître juste les instructions qu'on a besoin (for, if, comparaison, affines)
et également les instructions assync, finish, advance = barriere. On se donne un choix entre simuler une éxecution de programme paralléle et afficher un message si l'execution se bloquer
 ou iplementer les formules dans quadreplets de compilateur pour le dire direct si l'interblocage arrive lors de la compilation.
 L'atre idée et de visualiser les activités avec les barriéers.

\subsection{Planning}
21 - 27: installer X10 et essayer les programmes parallèles avec l'interblocage
 + écrire un compilateur avec if et for 
 
\section{Un peu de mise en forme}
Le retour à la ligne\\
doit être \newline
demandé 
explicitement.

Les lignes blanches ont une importance~: changement de paragraphe. \\

Un retour à la ligne suivi d'une ligne blanche commence un paragraphe détaché.

\noindent
On peut supprimer l'indentation.\\ 
On peut modifier le texte \textbf{Texte} \textit{Texte} \textsc{Text}.

\section{Listes et compagnie}
\begin{itemize}
 \item item 1
 \item item 2
 \item ...
\end{itemize}

\begin{enumerate}
 \item item 1
 \item item 2
 \item ...
\end{enumerate}

\begin{description}
 \item[un mot] une définition 
 \item[un autre mot] une autre définition 
 \item[encore un] ... 
\end{description}

\section{Les maths}
\label{section math}
Les équations peuvent être insérées dans le texte (exemple $x=2$), 
y compris si elles sont grosses (exemple $\sum_{i=0}^n \frac{x^i}{i!}$),
dans quel cas \texttt{displaystyle} peut servir (exemple $\displaystyle \sum_{i=0}^n \frac{x^i}{i!}$).

Elles peuvent être mises à part~:
\begin{equation}
a = \int_0^\infty f(u) du
\end{equation}
avec numérotation automatique
\begin{equation} \label{equation de b}
b = \iint f(t) dt
\end{equation}
ou sans numérotation~:
\begin{equation*}
c = \oint f(v) dv
\end{equation*}

\section{Les références}
Les sections et les environnements numérotés peuvent être référencés automatiquement grâce aux \texttt{label} et \texttt{ref}~:
les équations (\'Eq. \ref{equation de b} page \pageref{equation de b}), 
les sections (section \ref{section math} page \pageref{section math}), 
les environnements flottants.

\section{Les environnements flottants}
Ils permettent de placer automatiquement, d'ajouter des légendes et de référencer~: des tableaux (tab. \ref{mon tableau}), des figures, des algorithmes (Algo \ref{algo de redac}), etc.

\begin{table}
\begin{center}
\begin{tabular}{r|c|l|}
 texte & texte & texte \\
\hline
 texte plus long & texte plus long & texte plus long \\
\hline
\end{tabular}
\end{center}
\caption{je peux ajouter une légende \label{mon tableau}}
\end{table}

\begin{algorithm}
\While{j'ai des choses à dire}{
  je rédige \\
  \While{j'ai des erreurs de compilation}{
    je corrige mon code \LaTeX
  }
}
\caption{rédaction d'un rapport \label{algo de redac}} 
\end{algorithm}


% \addcontentsline{toc}{section}{Références}
% \bibliographystyle{alpha}
% \bibliography{../biblio.bib}

\end{document} 

